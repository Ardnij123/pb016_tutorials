\documentclass{beamer}

% chktex-file 1
% chktex-file 8

\usetheme[workplace=fi]{MU}
\usepackage[utf8]{inputenc}
\usepackage[
  main=czech,
  english
]{babel}
\usepackage[utf8]{inputenc}
\usepackage[T1]{fontenc}
\usepackage{csquotes}
\usepackage{booktabs}
\usepackage{amsmath}

\newcommand{\RR}{\ensuremath{\mathbb{R}}}

\usepackage{tikz}
\usetikzlibrary{arrows, shapes.geometric}

\newlength{\nodesize}
\setlength{\nodesize}{1.5em}
\tikzset{and/.style = {shape=circle,draw,minimum size=\nodesize}}
\tikzset{or/.style = {shape=rectangle,draw,minimum size=\nodesize}}

\tikzset{min/.style = {shape=circle,draw,minimum size=\nodesize}}
\tikzset{max/.style = {shape=rectangle,draw,minimum size=\nodesize}}
\tikzset{rand/.style = {diamond,draw,minimum size=\nodesize}}

\tikzset{edge/.style = {->,> = latex'}}

% Allows for split by delimiter
% \getNth{string}{delimiter}{entry_no}
% Source: https://tex.stackexchange.com/questions/12810/how-do-i-split-a-string
\ExplSyntaxOn
\NewDocumentCommand{\getNth}{mmm}
  {
    % #1 string, #2 separator, #3 index
    \seq_set_split:Nnx \l_tmpa_seq { #2 } { #1 }
    \seq_item:Nn \l_tmpa_seq { #3 }
  }
\ExplSyntaxOff

\setbeamertemplate{enumerate item}{\alph{enumi})}  % chktex 9 chktex 10

\newcommand{\tictactoe}[2]{%
    \draw (#1.center) ++(-.5,-.5) -- +(0,1);
    \draw (#1.center) ++(-.1667,-.5) -- +(0,1);
    \draw (#1.center) ++(.1667,-.5) -- +(0,1);
    \draw (#1.center) ++(.5,-.5) -- +(0,1);
    \draw (#1.center) ++(-.5,-.5) -- +(1,0);
    \draw (#1.center) ++(-.5,-.1667) -- +(1,0);
    \draw (#1.center) ++(-.5,.1667) -- +(1,0);
    \draw (#1.center) ++(-.5,.5) -- +(1,0);
    \draw (#1.center) ++(-.3333,.3333) node {\getNth{#2}{;}{1}};
    \draw (#1.center) ++(0,.3333) node {\getNth{#2}{;}{2}};
    \draw (#1.center) ++(.3333,.3333) node {\getNth{#2}{;}{3}};
    \draw (#1.center) ++(-.3333,0) node {\getNth{#2}{;}{4}};
    \draw (#1.center) ++(0,0) node {\getNth{#2}{;}{5}};
    \draw (#1.center) ++(.3333,0) node {\getNth{#2}{;}{6}};
    \draw (#1.center) ++(-.3333,-.3333) node {\getNth{#2}{;}{7}};
    \draw (#1.center) ++(0,-.3333) node {\getNth{#2}{;}{8}};
    \draw (#1.center) ++(.3333,-.3333) node {\getNth{#2}{;}{9}};
}

\title{Výroková logika}
\author{Jindřich Matuška}
\institute[FI MU]{Faculty of Informatics, Masaryk University}
\date{24. října 2024}

\AtBeginSection[]
{
    \begin{frame}
        \frametitle{Obsah}
        \tableofcontents[currentsection]
    \end{frame}
}

\begin{document}

\frame{\titlepage}

\frame{
    \frametitle{Čas na odpovědníky}
}

\begin{frame}
    \frametitle{Obsah}
    \tableofcontents
\end{frame}

\begin{section}{Syntax}

\frame{
    \frametitle{Syntax výrokové logiky}
    \begin{itemize}
        \item Abeceda
            \begin{itemize}
                \item Výrokové proměnné $\mathcal{P} = \{p, q, r, \ldots\}$
                \item Logické spojky $\neg, \vee, \wedge, \implies, \iff, \ldots$
                \item Pomocné symboly závorek $(, )$
            \end{itemize}
        \item Formule výrokové logiky
            \begin{itemize}
                \item Každá proměnná $p\in \mathcal{P}$ je formule
                \item Jsou-li $p, q$ formule, pak jsou formule i
                    $\neg (p), (p)\vee(q), (p)\wedge(q), (p)\implies(q), (p)\iff(q), \ldots$
                \item Množinu formulí výrokové logiky značíme $\mathcal{F}$
            \end{itemize}
    \end{itemize}
}

\end{section}

\begin{section}{Sémantika}

\frame{
    \frametitle{Sémantika výrokové logiky}

    \begin{itemize}
        \item Interpretace $I$
            \begin{itemize}
                \item Zobrazení $I: \mathcal P\to \{0,1\}$
                \item Přiřazuje výrokovým proměnným 1 (pravda) či 0 (nepravda)
            \end{itemize}
        \item Valuace příslušící $I$
            \begin{itemize}
                \item Zobrazení $I: \mathcal F\to \{0,1\}$
                \item Rozšíření interpretace na všechny výrokové proměnné podle sémantiky spojek
            \end{itemize}
    \end{itemize}
}

\frame{
    \frametitle{Pojmy}

    \begin{itemize}
        \item Formule $\varphi$ pravdivá v interpretaci $I$ --- $I(\varphi) = 1$
        \item Formule $\varphi$ logicky pravdivá (tautologie) --- $\forall I.\ I(\varphi) = 1$
        \item Kontradikce --- $\forall I.\ I(\varphi) = 0$
        \item Splnitelná --- $\exists I.\ I(\varphi) = 1$
        \item Sémanticky ekvivalentní formule $\varphi, \psi$ --- $\forall I.\ I(\varphi) = I(\psi)$
    \end{itemize}
}

\frame{
    \frametitle{Pravdivostní tabulka}

    \begin{center}
    \begin{tabular}{cc | ccccc}  % chktex 44
        $p$ & $q$ & $p$ & $\wedge$ & $(q$ & $\implies$ & $\neg p)$ \\ \midrule
        $0$ & $0$ & $0$ & $0$ & $0$ & $1$ & $1$\\
        $0$ & $1$ & $0$ & $0$ & $1$ & $1$ & $1$\\
        $1$ & $0$ & $1$ & $1$ & $0$ & $1$ & $0$\\
        $1$ & $1$ & $1$ & $0$ & $1$ & $0$ & $0$
    \end{tabular}
    \end{center}
}

\frame{
	\frametitle{Příklad 5.2.4.}
    Mějme formuli výrokové logiky $\varphi \equiv (p\wedge q) \iff (\neg q \wedge r)$.
    Bez použití pravdivostní tabulky určete:
    \begin{itemize}
        \item zda je $\varphi$ pravdivá v interpretaci $I(p) = 0, I(q) = I(r) = 1$,
        \item zda je $\varphi$ kontradikcí,
        \item zda je $\varphi$ tautologií,
        \item zda je $\varphi$ splnitelná; případně nalezněte nějaký její model.
    \end{itemize}
}

\frame{
    \frametitle{Příklad 5.2.5.}
    Udejte příklad formule $\varphi$ takové, že:
    \begin{itemize}
        \item $\varphi$ obsahuje právě 3 různé výrokové proměnné a je pravdivá právě ve 3 interpretacích,
        \item $\varphi$ obsahuje právě 3 různé výrokové proměnné, je pravdivá právě ve 3 interpretacích a obsahuje pouze logickou spojku $\implies$,
        \item $\varphi$ je kontradikce, obsahuje právě 2 různé výrokové proměnné, každou dvakrát
    \end{itemize}
    (Uvažujte interpretaci jako zobrazení přiřazující hodnoty právě výrokovým proměnným vyskytujícím se v $\varphi$.)
}

\end{section}

\begin{section}{Normální formy}

\frame{
    \frametitle{Normální formy}

    \begin{itemize}
        \item Literál je výroková proměnná nebo její negace
        \item Klauzule je disjunkce literálů ($\vee$)
        \item Duální klauzule je konjunkce literálů ($\wedge$)
    \end{itemize}
    \vspace{1em}
    \begin{itemize}
        \item Konjunktivní normální forma (KNF) --- konjunkce klauzulí
        \item Disjunktivní normální forma (DNF) --- disjunkce duálních klauzulí
        \item Úplná normální forma (ÚKNF, ÚDNF) --- každá klauzule obsahuje každou výrokovou proměnnou právě jednou
    \end{itemize}
}

\frame{
    \frametitle{Příklad 5.3.2, 5.3.3}

    Použitím ekvivalentních úprav nalezněte KNF následující formule.
    \begin{itemize}
        \item $\varphi \equiv (p\implies q)\implies r$
    \end{itemize}

    Použitím ekvivalentních úprav nalezněte DNF následující formule.
    \begin{itemize}
        \item $\varphi \equiv (p\iff q)\implies (r\vee s)$
    \end{itemize}
}

\end{section}

\begin{section}{Množiny formulí, splnitelnost, vyplývání}

\frame{
    \frametitle{Splnitelnost množiny formulí}

    Množina formulí $\mathcal T$ je splnitelná,
    jestliže existuje interpretace $I$,
    v níž jsou pravdivé všechny formule $\varphi\in\mathcal T$.
    O množině $\mathcal T$ řekneme, že je pravdivá v interpretaci $I$
    a tuto interpretaci nazveme modelem množiny $\mathcal T$.
    \vspace{1em}

    Formule $\varphi$ logicky vyplývá z množiny formulí $\mathcal T$
    ($\mathcal T \models \varphi$),
    právě když je $\varphi$ pravdivá v každém modelu množiny $\mathcal T$.
}

\frame{
    \frametitle{Příklad 5.4.1, 5.4.2}

    Určete splnitelnost následujících množin formulí. Je-li množina splnitelná,
    nalezněte nějaký její model. Vyhněte se použití pravdivostních tabulek.

    \begin{itemize}
        \item $\mathcal T = \{(p\implies q)\wedge r, q\wedge r, r\implies s, p\wedge \neg s\}$
    \end{itemize}
    \vspace{1em}

    Je množina formulí $\mathcal T = \emptyset$ splnitelná? Dokažte.
}

\frame{
    \frametitle{Příklad 5.4.7}

    Uvažujte množinu formulí výrokové logiky $\mathcal T$ a formule $\varphi, \psi$.
    Rozhodněte o platnosti následujících tvzení:

    \begin{itemize}
        \item Pokud $\mathcal T \models \varphi$ a $\mathcal T\models \psi$,
            pak $\mathcal T \models \varphi \vee \psi$
        \item Pokud $\mathcal T \models \varphi$ a $\mathcal T\models \psi$,
            pak $\mathcal T \models \varphi \wedge \psi$
        \item Pokud $\mathcal T \models \varphi\vee \psi$,
            pak $\mathcal T \models \varphi$ nebo $\mathcal T \models \psi$
        \item Pokud $\mathcal T \models \varphi\wedge \psi$,
            pak $\mathcal T \models \varphi$ a $\mathcal T \models \psi$
        \item Pokud $\mathcal T \models \varphi$,
            pak $\mathcal T \models \varphi \vee \psi$
        \item Pokud $\mathcal T \models \varphi$,
            pak $\mathcal T \models \varphi \wedge \psi$
        \item Pokud $\mathcal T \models \varphi$,
            pak $\mathcal T \cup \{\psi\} \models \varphi$
        \item Pokud $\mathcal T \cup \{\psi\} \models \varphi$,
            pak $\mathcal T \models \varphi$
    \end{itemize}
}

\frame{
    \frametitle{Příklad 5.4.8}

    Ukažte, že logické vyplývání $\mathcal T \models \varphi$ platí,
    právě když množina $\mathcal T \cup \{\neg \varphi\}$ je nesplnitelná.
}

\end{section}

\begin{section}{Logické spojky}

\frame{
    \frametitle{Sémantika logické spojky}

    Sémantika $n$-ární logické spojky $\square$ je funkce $f_\square: {\{0, 1\}}^n\to \{0,1\}$

    \vspace{1em}
    Valuace formule $\psi\equiv \square(\varphi_1, \ldots, \varphi_n)$ v interpretaci $I$
    je dána:
    \begin{equation*}
        I(\psi) = f_{\square}(I(\varphi_1), \ldots, I(\varphi_n))
    \end{equation*}
    \vspace{1em}

    Sémantika logické spojky lze definovat např.\ pomocí tabulky
    nebo již definovaných spojek
}

\frame{
    \frametitle{Úplný systém logických spojek}

    Množina logických spojek tvoří úplný systém logických spojek,
    pokud lze formulemi obsahujícími pouze spojky z dané množiny
    vyjádřit libovolnou logickou funkci.

    \vspace{1em}
    Například množina $\{\wedge, \vee, \neg, \implies, \iff\}$
    je úplný systém logických spojek.
}

\frame{
    \frametitle{Příklad 5.5.1}

    Kolik existuje různých vzájemně neekvivalentních $n$-árních spojek?
}

\frame{
    \frametitle{Příklad 5.5.5 a)}  % chktex 9

    Ukažte, že $\{NOR\}$ je úplný systém logických spojek.
    Vyjděte z předpoklad, že $\{\implies, \neg\}$ je úplný systém logických spojek.\\
    (Tj.\ vyjádřete formule $\varphi \implies \psi$ a $\neg \varphi$
    pomocí formulí obsahujících pouze spojku $NOR$).
    \begin{equation*}
        \varphi NOR \psi \equiv \neg(\varphi \vee \psi)
    \end{equation*}
}  % chktex 10

\end{section}

\end{document}  % chktex 17
