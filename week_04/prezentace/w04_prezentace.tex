\documentclass{beamer}

% chktex-file 1
% chktex-file 8

\usetheme[workplace=fi]{MU}
\usepackage[utf8]{inputenc}
\usepackage[
  main=czech,
  english
]{babel}
\usepackage[utf8]{inputenc}
\usepackage[T1]{fontenc}
\usepackage{csquotes}
\usepackage{booktabs}
\usepackage{amsmath}

\newcommand{\RR}{\ensuremath{\mathbb{R}}}

\usepackage{tikz}
\usetikzlibrary{arrows}

\newlength{\nodesize}
\setlength{\nodesize}{1.5em}
\tikzset{and/.style = {shape=circle,draw,minimum size=\nodesize}}
\tikzset{or/.style = {shape=rectangle,draw,minimum size=\nodesize}}
\tikzset{edge/.style = {->,> = latex'}}

% Allows for split by delimiter
% \getNth{string}{delimiter}{entry_no}
% Source: https://tex.stackexchange.com/questions/12810/how-do-i-split-a-string
\ExplSyntaxOn
\NewDocumentCommand{\getNth}{mmm}
  {
    % #1 string, #2 separator, #3 index
    \seq_set_split:Nnx \l_tmpa_seq { #2 } { #1 }
    \seq_item:Nn \l_tmpa_seq { #3 }
  }
\ExplSyntaxOff

\newcommand{\tictactoe}[2]{%
    \draw (#1.center) ++(-.5,-.5) -- +(0,1);
    \draw (#1.center) ++(-.1667,-.5) -- +(0,1);
    \draw (#1.center) ++(.1667,-.5) -- +(0,1);
    \draw (#1.center) ++(.5,-.5) -- +(0,1);
    \draw (#1.center) ++(-.5,-.5) -- +(1,0);
    \draw (#1.center) ++(-.5,-.1667) -- +(1,0);
    \draw (#1.center) ++(-.5,.1667) -- +(1,0);
    \draw (#1.center) ++(-.5,.5) -- +(1,0);
    \draw (#1.center) ++(-.3333,.3333) node {\getNth{#2}{;}{1}};
    \draw (#1.center) ++(0,.3333) node {\getNth{#2}{;}{2}};
    \draw (#1.center) ++(.3333,.3333) node {\getNth{#2}{;}{3}};
    \draw (#1.center) ++(-.3333,0) node {\getNth{#2}{;}{4}};
    \draw (#1.center) ++(0,0) node {\getNth{#2}{;}{5}};
    \draw (#1.center) ++(.3333,0) node {\getNth{#2}{;}{6}};
    \draw (#1.center) ++(-.3333,-.3333) node {\getNth{#2}{;}{7}};
    \draw (#1.center) ++(0,-.3333) node {\getNth{#2}{;}{8}};
    \draw (#1.center) ++(.3333,-.3333) node {\getNth{#2}{;}{9}};
}

\title{Dekompozice problému,\\problémy s omezujícími podmínkami}
\author{Jindřich Matuška}
\institute[FI MU]{Faculty of Informatics, Masaryk University}
\date{17. října 2024}

% 3 Dekompozice problému, Problémy s omezujícími podmínkami
    % 3.1 Dekompozice, AND/OR grafy
        % Definice/pojmy a značení AND/OR grafu
        % Příklad AND/OR grafu
            % Piškvorky (střídání 2 hráčů)
            % Bludiště/Prohledávání (pouze OR uzly)
        % Strom řešení
            % Splněný vrchol ve stromu řešení
        % Ukázat strom řešení na příkladu
            % Piškvorky (střídání 2 hráčů)
            % Bludiště/Prohledávání (pouze OR uzly)
        % P3.1.2 Libovolný AND/OR <=> Bipartitní
        % P3.1.4 Splnitelnost grafu
            % Proč není graf s 6-7 stromem řešení?
        % P3.1.6 AND/OR graf piškvorek

    % Jupyter lab
        % E1 Dragon problem
        % E1.1 Vytvoření AND/OR grafu
        % E1.2-3 Ohodnocení splnitelnosti
        % E1.4 Výpis řešení

    % 3.1 Problémy s omezujícími podmínkami
        % Kecy o CSP
        % Příklad Typová inference
        % Definice Problému s omezujícími podmínkami
        % Příklad sudoku
            % TODO: udělat Clingo program
        % P3.2.1 Stav grafů, popište řešení
        % P3.2.3

\AtBeginSection[]
{
    \begin{frame}
        \frametitle{Obsah}
        \tableofcontents[currentsection]
    \end{frame}
}

\begin{document}

\frame{\titlepage}

\frame{
    \frametitle{Čas na odpovědníky}
}

\begin{frame}
    \frametitle{Obsah}
    \tableofcontents
\end{frame}

\begin{section}{Dekompozice, AND/OR grafy}

\frame{
    \frametitle{AND/OR graf}
    Syntax
    \begin{itemize}
        \item orientovaný graf
        \item vnitřní vrcholy typu \textbf{AND} nebo \textbf{OR}
            \begin{itemize}
                \item značíme kolečkem (AND) či čtverečkem (OR)
            \end{itemize}
        \item koncové vrcholy s přiřazenou hodnotou
    \end{itemize}
}

\frame{
    \frametitle{Piškvorky}
    \vspace{-2em}
\begin{center}
    \setlength{\nodesize}{4em}
    \begin{tikzpicture}
        \node[or] (a) at (0,0) {};
        \tictactoe{a}{;;o;o;x;o;x;;x}

        \node[and] (b) at (-4,-2) {};
        \draw[edge] (a) to (b);
        \tictactoe{b}{o;;o;o;x;o;x;;x}

        \node[and] (c) at (0,-2) {};
        \draw[edge] (a) to (c);
        \tictactoe{c}{;o;o;o;x;o;x;;x}

        \node[and] (d) at (4,-2) {};
        \draw[edge] (a) to (d);
        \tictactoe{d}{;;o;o;x;o;x;o;x}

        \node[or] (e) at (-5,-4) {};
        \draw[edge] (b) to (e);
        \tictactoe{e}{o;x;o;o;x;o;x;;x}

        \node[or] (f) at (-3,-4) {};
        \draw[edge] (b) to (f);
        \tictactoe{f}{o;;o;o;x;o;x;x;x}

        \node[or] (g) at (-1,-4) {};
        \draw[edge] (c) to (g);
        \tictactoe{g}{x;o;o;o;x;o;x;;x}

        \node[or] (h) at (1,-4) {};
        \draw[edge] (c) to (h);
        \tictactoe{h}{;o;o;o;x;o;x;x;x}

        \node[or] (i) at (3,-4) {};
        \draw[edge] (d) to (i);
        \tictactoe{i}{x;;o;o;x;o;x;o;x}

        \node[or] (j) at (5,-4) {};
        \draw[edge] (d) to (j);
        \tictactoe{j}{;x;o;o;x;o;x;o;x}

        \node[and] (k) at (-5,-6) {};
        \draw[edge] (e) to (k);
        \tictactoe{k}{o;x;o;o;x;o;x;o;x}

        \node[and] (l) at (5,-6) {};
        \draw[edge] (j) to (l);
        \tictactoe{l}{o;x;o;o;x;o;x;o;x}
    \end{tikzpicture}
\end{center}
}

\frame{
    \frametitle{Bludiště (vyhledávací strom)}
    \hfill
    \begin{tikzpicture}
        \node (a) at (0,4) {a};
        \node (ab) at (0,1) {};
        \node (e) at (0,0) {e};
        \node (hjj) at (1,4) {};
        \node (hhj) at (1,2) {};
        \node (b) at (1,1) {b};
        \node (c) at (1,0) {c};
        \node (j) at (2,4) {j};
        \node (hhi) at (2,3) {};
        \node (h) at (2,2) {h};
        \node (hii) at (3,4) {};
        \node (hi) at (3,3) {};
        \node (f) at (3,0) {f};
        \draw (f) circle(.3);
        \node (hiii) at (4,4) {};
        \node (i) at (4,3) {i};
        \draw (i) circle(.3);
        \node (hd) at (4,2) {};
        \node (d) at (4,1) {d};
        \node (g) at (4,0) {g};
        \draw (a) -- (ab.center);
        \draw (ab.center) -- (b);
        \draw (e) -- (c);
        \draw (b) -- (c);
        \draw (hhj.center) -- (hjj.center);
        \draw (hjj.center) -- (j);
        \draw (hhj.center) -- (h);
        \draw (b) -- (d);
        \draw (c) -- (f);
        \draw (hhi.center) -- (h);
        \draw (hd.center) -- (h);
        \draw (hhi.center) -- (hi.center);
        \draw (hi.center) -- (hii.center);
        \draw (hiii.center) -- (hii.center);
        \draw (hiii.center) -- (i);
        \draw (hd.center) -- (d);
        \draw (g) -- (d);
    \end{tikzpicture}
    \hfill
    \begin{tikzpicture}
        \node[or] (a) at (0, 0){a};
        \node[or] (b) at (0, -1){b};
        \draw[edge] (a) to (b);
        \node[or] (c) at (-1.5,-2){c};
        \draw[edge] (b) to (c);
        \node[or] (d) at (1.5, -2){d};
        \draw[edge] (b) to (d);
        \node[or] (e) at (-2, -3){e};
        \draw[edge] (c) to (e);
        \node[and] (f) at (-1, -3){f};
        \draw[edge] (c) to (f);
        \node[or] (g) at (1, -3){g};
        \draw[edge] (d) to (g);
        \node[or] (h) at (2, -3){h};
        \draw[edge] (d) to (h);
        \node[and] (i) at (1.5, -4){i};
        \draw[edge] (h) to (i);
        \node[or] (j) at (2.5, -4){j};
        \draw[edge] (h) to (j);
    \end{tikzpicture}
    \hfill\phantom{}
}

\frame{
    \frametitle{Strom řešení, Splnitelnost}
    Strom řešení $T$ problému $P$ s AND/OR grafem $G$
    \begin{itemize}
        \item podgraf grafu $T\subseteq G$, $T$ je strom
        \item kořen $T$ je vrchol reprezentující $P$
        \item pro uzel $T$ typu AND jsou všichni jeho následovníci z $G$ v $T$
        \item pro uzel $T$ typu OR je právě jeden jeho následovník z $G$ v $T$
    \end{itemize}
    \vspace{1em}
    Uzel je splněný tehdy, když pro něj existuje strom řešení se všemi koncovými vrcholy nastavenými na \textit{true}.
}

\frame{
    \frametitle{Strom řešení, Splnitelnost}
    Je uzel $1$ splnitelný? Kolik má stromů řešení, které jej splňují?
    \begin{center}
    \begin{tikzpicture}
        \node[or] (1) at (0, 0) {1};
        \node[and] (2) at (-1, -1) {2};
        \node[or] (3) at (0, -1) {3};
        \node (f1) at (1, -1) {false};
        \node[and] (4) at (-1, -2) {4};
        \node[and] (5) at (0, -2) {5};
        \node[or] (6) at (1, -2) {6};
        \node (t) at (0, -3) {true};
        \node (f2) at (2, -3) {false};
        \draw[edge] (1) to (2);
        \draw[edge] (1) to (3);
        \draw[edge] (1) to (f1);
        \draw[edge] (2) to (4);
        \draw[edge] (2) to (5);
        \draw[edge] (3) to (5);
        \draw[edge] (3) to (6);
        \draw[edge] (4) to (t);
        \draw[edge] (5) to (t);
        \draw[edge] (6) to (t);
        \draw[edge] (6) to (f2);
    \end{tikzpicture}
    \end{center}
    \pause

    Je splnitelný, splňují jej indukované grafy na vrcholech
    \begin{equation*}
        \{1, 2, 4, 5, true\}, \{1, 3, 5, true\}, \{1, 3, 6, true\}
    \end{equation*}
}

\frame{
    \frametitle{Příklad 3.1.2}
    Ukažte, že každý AND/OR graf lze převést na ekvivalentní bipartitní
    AND/OR graf, ve kterém jsou následníky vrcholů typu AND vždy vrcholy typu OR a opačně.
    Jaké to má praktické důsledky pro implementaci AND/OR grafů?
    \pause
    \vspace{1em}

    Pokud jsou v grafu nějaké uzly stejného typu spojené hranou,
    pak je tento graf ekvivalentní s grafem, ve kterém jsou tyto uzly
    nahrazeny jedním uzlem, který má výstupní hrany do všech uzlů,
    které jsou sjednocením potomků těchto uzlů.
    \vspace{1em}
    \begin{equation*}
        (P \wedge A) \wedge (B \iff P) \iff A\wedge B
    \end{equation*}
    \begin{equation*}
        (P \vee A) \vee (B \iff P) \iff A\vee B
    \end{equation*}
}

\frame{
    \frametitle{Příklad 3.1.4}
    Rozhodněte, zda je počáteční uzel, značené 1, splněn v následujícím
    AND/OR grafu. Spočtěte také, kolik takových
    stromů řešení existujei.

    \begin{center}
    \begin{tikzpicture}
        \node[and] (1) at (0,0) {1};
        \node[or] (2) at (0,-1) {2};
        \draw[edge] (1) to (2);
        \node[and] (3) at (1,-1) {3};
        \draw[edge] (1) to (3);
        \node[and] (4) at (0,-2) {4};
        \draw[edge] (2) to (4);
        \node[or] (5) at (1,-2) {5};
        \draw[edge] (3) to (5);
        \node[and] (6) at (-1,-2) {6};
        \draw[edge] (2) to (6);
        \node[and] (7) at (-2,-2) {7};
        \draw[edge] (6) to[bend left] (7);
        \draw[edge] (7) to[bend left] (6);
        \node (t) at (0,-3) {true};
        \draw[edge] (3) to (t);
        \draw[edge] (4) to (t);
        \draw[edge] (5) to (t);
        \node (f) at (1,-3) {false};
        \draw[edge] (5) to (f);
    \end{tikzpicture}
    \end{center}
    \pause

    Proč není některé řešení zahrnující uzly 6, 7 splňující?
}

\frame{
    \frametitle{Příklad 3.1.6}
    Uvažte rozehranou partii piškvorek zobrazenou na
    obrázku, v níž je právě na tahu hráč kreslící kolečka. Sestrojte AND/OR
    graf pro tuto partii. Nalezněte stromy řešení tohoto AND/OR grafu a
    interpretujte jejich vztah k výsledku partie.

    \begin{center}
    \begin{tikzpicture}[scale=2]
        \node (a) at (0, 0) {};
        \tictactoe{a}{o;x;o;o;x;o;x;o;x}
    \end{tikzpicture}
    \end{center}
}

\frame{
    \frametitle{Příklad 3.1.6}
    \begin{center}
        \setlength{\nodesize}{4em}
    \begin{tikzpicture}
        \node[or] (a) at (0, 0) {};
        \tictactoe{a}{x;;o;;x;;x;o;o}

        \node[and] (b) at (-2, -2) {};
        \tictactoe{b}{x;;o;o;x;;x;o;o}
        \draw[edge] (a) to (b);

        \node[and] (c) at (0, -2) {};
        \tictactoe{c}{x;;o;;x;o;x;o;o}
        \draw[edge] (a) to (c);

        \node[and] (d) at (2, -2) {};
        \tictactoe{d}{x;o;o;;x;;x;o;o}
        \draw[edge] (a) to (d);

        \node[or] (e) at (-3, -4) {};
        \tictactoe{e}{x;x;o;o;x;;x;o;o}
        \draw[edge] (b) to (e);

        \node[or] (f) at (-1, -4) {};
        \tictactoe{f}{x;;o;o;x;x;x;o;o}
        \draw[edge] (b) to (f);

        \node[or] (g) at (1, -4) {};
        \tictactoe{g}{x;o;o;x;x;;x;o;o}
        \draw[edge] (d) to (g);

        \node[or] (h) at (3, -4) {};
        \tictactoe{h}{x;o;o;;x;x;x;o;o}
        \draw[edge] (d) to (h);

        \node[and] (i) at (-3, -6) {};
        \tictactoe{i}{x;x;o;o;x;o;x;o;o}
        \draw[edge] (e) to (i);

        \node[or] (j) at (-1, -6) {};
        \tictactoe{j}{x;o;o;o;x;x;x;o;o}
        \draw[edge] (f) to (j);

        \node[or] (k) at (3, -6) {};
        \tictactoe{k}{x;o;o;o;x;x;x;o;o}
        \draw[edge] (h) to (k);
    \end{tikzpicture}
    \end{center}
}

\end{section}

\begin{section}{Jupyter lab 1}
\frame{
    \frametitle{Jupyter lab 1 - Dragon problem}
    \begin{itemize}
        \item Reprezentace problému jako AND/OR graf
        \item Sémantika AND a OR uzlů
        \item Výpis plánu řešení
    \end{itemize}
    }
\end{section}

\begin{section}{Problémy s omezujícími podmínkami}

\frame{
    \frametitle{Problémy s omezujícími podmínkami}
    \begin{itemize}
        \item CSP --- Constraint satisfaction problem
        \item Deklarativní programování
        \item Definujeme očekávané vlastnosti řešení
        \vspace{1em}
        \item Obarvení grafu
        \item Algebrogram
        \item Problém N dam
    \end{itemize}
}

\frame{
    \frametitle{Odvození typů}
    U následujícího programu poveďte typovou inferenci (tj.\ odvození typů) funkcí
    funkcí f a g. Neuvažujte přetížení, funkce budou mít vždy jeden typ.

    \vspace{1em}
    \texttt{%
        x = f(g(0))\\
        x = f(x)\\
        x = g(x)\\
    }
}

\frame{
    \frametitle{Problém s omezujícími podmínkami}
    \begin{itemize}
        \item soubor proměnných $X_1, \ldots, X_n$ s neprázdnými doménami $D_1, \ldots, D_n$
        \item soubor omezení $C_1, \ldots, C_m$;
            každé omezení je podmnožinou $D_1\times \cdots \times D_n$
        \item někdy navíc účelová funkce $f:D_1\times \cdots \times D_n \to \RR$
            \vspace{1em}
        \item Řešením je $n$-tice $(x_1, \ldots, x_n)$, která splňuje všechna omezení
        \item $(x_1, \ldots, x_n) \in \bigcap_i\in \{1, \ldots, m\}C_i$
            \vspace{1em}
        \item Pokud je řešení více, můžeme hledat takové, které maximalizuje (či minimalizuje) účelovou funkci
    \end{itemize}
}

% TODO: sem vložit kód sudoku

\frame{
    \frametitle{Graf stavů}
    \begin{itemize}
        \item Stromová struktura
        \item Každý uzel přiřazuje hodnotu jedné proměnné
        \item List (Koncový uzel) pokud již nelze rozšířit bez porušení pravidel
        \item Cílová podmínka --- přiřazení je úplné
    \end{itemize}
}

\frame{
    \frametitle{Příklad 3.2.1}
    Sestavte graf stavů pro následující CSP.\@ Proměnné přiřazujte v sekvenčním
    pořadí podle jejich deklarace. Popište řešení.

    \vspace{1em}
    \texttt{%
        A in 2..4\\
        B in 2..3\\
        C in 0..6\\
        \vspace{1em}
        A - B >= C\\
        A * (B-1) != B + C\\  % chktex 26
        A != B  % chktex 26
    }
}

\frame{
    \frametitle{Příklad 3.2.3}
    Jako kouzelný čtverec označíme čtvercovou matici, kde součet čísel na
    každém řádku a v každém sloupci je stejný. Jednotlivé buňky mohou nabývat celočíselných
    hodnot mezi 1 a $n^2$, kde $n$ je dimenze uvažované matice.

    Jedním z netriviálních řešení pro $n = 3$ je kouzelný čtverec níže.
    
    Formulujte tento problém jako CSP pro uvedené $n = 3$. Kromě omezení
    nezapomeňte uvést význam zavedených proměnných a jejich domény.
    
    \begin{center}
        \begin{tikzpicture}[scale=1.5]
        \node (a) at (0, 0) {};
        \tictactoe{a}{2;9;4;7;5;3;6;1;8}
    \end{tikzpicture}
    \end{center}
}

\end{section}

\begin{section}{Jupyter lab 2}
\frame{
    \frametitle{Jupyter lab 2 --- Constraint satisfaction problems}

    \begin{itemize}
        \item Seznámení s knihovnou \texttt{constraint}
        \item Map coloring problem
        \item Problém 8 dam
    \end{itemize}
}
\end{section}

\end{document}
