\documentclass{beamer}

% chktex-file 1
% chktex-file 8
% chktex-file 9
% chktex-file 10

\usetheme[workplace=fi]{MU}
\usepackage[utf8]{inputenc}
\usepackage[
  main=czech,
  english
]{babel}
\usepackage[utf8]{inputenc}
\usepackage[T1]{fontenc}
\usepackage{csquotes}
\usepackage{booktabs}
\usepackage{amsmath}
\usepackage{multicol}
\usepackage{listings}

\newcommand{\RR}{\ensuremath{\mathbb{R}}}

\usepackage{tikz}
\usetikzlibrary{arrows, shapes.geometric}
\usetikzlibrary{calc, positioning}
\usetikzlibrary{decorations.pathmorphing, decorations.pathreplacing, decorations.shapes}
\usepackage{pgfplots}

\newlength{\nodesize}
\setlength{\nodesize}{1.5em}
\tikzset{and/.style = {shape=circle,draw,minimum size=\nodesize}}
\tikzset{or/.style = {shape=rectangle,draw,minimum size=\nodesize}}

\tikzset{min/.style = {shape=circle,draw,minimum size=\nodesize}}
\tikzset{max/.style = {shape=rectangle,draw,minimum size=\nodesize}}
\tikzset{rand/.style = {diamond,draw,minimum size=\nodesize}}

\tikzset{edge/.style = {->,> = latex'}}

% Allows for split by delimiter
% \getNth{string}{delimiter}{entry_no}
% Source: https://tex.stackexchange.com/questions/12810/how-do-i-split-a-string
\ExplSyntaxOn
\NewDocumentCommand{\getNth}{mmm}
  {
    % #1 string, #2 separator, #3 index
    \seq_set_split:Nnx \l_tmpa_seq { #2 } { #1 }
    \seq_item:Nn \l_tmpa_seq { #3 }
  }
\ExplSyntaxOff

\setbeamertemplate{enumerate item}{\alph{enumi})}  % chktex 9 chktex 10

\newcommand{\tictactoe}[2]{%
    \draw (#1.center) ++(-.5,-.5) -- +(0,1);
    \draw (#1.center) ++(-.1667,-.5) -- +(0,1);
    \draw (#1.center) ++(.1667,-.5) -- +(0,1);
    \draw (#1.center) ++(.5,-.5) -- +(0,1);
    \draw (#1.center) ++(-.5,-.5) -- +(1,0);
    \draw (#1.center) ++(-.5,-.1667) -- +(1,0);
    \draw (#1.center) ++(-.5,.1667) -- +(1,0);
    \draw (#1.center) ++(-.5,.5) -- +(1,0);
    \draw (#1.center) ++(-.3333,.3333) node {\getNth{#2}{;}{1}};
    \draw (#1.center) ++(0,.3333) node {\getNth{#2}{;}{2}};
    \draw (#1.center) ++(.3333,.3333) node {\getNth{#2}{;}{3}};
    \draw (#1.center) ++(-.3333,0) node {\getNth{#2}{;}{4}};
    \draw (#1.center) ++(0,0) node {\getNth{#2}{;}{5}};
    \draw (#1.center) ++(.3333,0) node {\getNth{#2}{;}{6}};
    \draw (#1.center) ++(-.3333,-.3333) node {\getNth{#2}{;}{7}};
    \draw (#1.center) ++(0,-.3333) node {\getNth{#2}{;}{8}};
    \draw (#1.center) ++(.3333,-.3333) node {\getNth{#2}{;}{9}};
}

\tikzset{basic/.style={draw,fill=blue!20,text badly centered}}
\tikzset{neurons/.style={basic,circle,fill=blue!10}}
\tikzset{input/.style={basic,circle}}
\tikzset{weights/.style={basic,fill=white,rectangle}}

\newcommand{\nndemo}{
\begin{center}
\scalebox{0.9}{
\begin{tikzpicture}[scale=0.4]
    \node[neurons] (o2) {$y_2$};
    \node[right of=o2] (u2) {};
        \path[draw,->] (o2) -- (u2);
    \node[neurons,above=2em of o2] (o1) {$y_1$};
    \node[right of=o1] (u1) {};
        \path[draw,->] (o1) -- (u1);
    \node[below=2em of o2] (odots) {$\vdots$};
    \node[neurons,below=2em of odots] (om) {$y_m$};
    \node[right of=om] (um) {};
        \path[draw,->] (om) -- (um);
    
    \node[left=5em of o2] (b) {};
    \node[neurons,above=1em of b] (b2) {};
        \path[draw,->] (b2) -- (o1);
        \path[draw,->] (b2) -- (o2);
        \path[draw,->] (b2) -- (om);
    \node[neurons,above=2em of b2] (b1) {};
        \path[draw,->] (b1) -- (o1);
        \path[draw,->] (b1) -- (o2);
        \path[draw,->] (b1) -- (om);
    \node[neurons,below=2em of b2] (b3) {};
        \path[draw,->] (b3) -- (o1);
        \path[draw,->] (b3) -- (o2);
        \path[draw,->] (b3) -- (om);
    \node[below=2em of b3] (bdots) {$\vdots$};
    \node[neurons,below=2em of bdots] (bk) {};
        \path[draw,->] (bk) -- (o1);
        \path[draw,->] (bk) -- (o2);
        \path[draw,->] (bk) -- (om);
    
    \node[left=5em of b] (a) {};
    \node[neurons,above=1em of a] (a2) {};
        \path[draw,->] (a2) -- (b1);
        \path[draw,->] (a2) -- (b2);
        \path[draw,->] (a2) -- (b3);
        \path[draw,->] (a2) -- (bk);
    \node[neurons,above=2em of a2] (a1) {};
        \path[draw,->] (a1) -- (b1);
        \path[draw,->] (a1) -- (b2);
        \path[draw,->] (a1) -- (b3);
        \path[draw,->] (a1) -- (bk);
    \node[neurons,below=2em of a2] (a3) {};
        \path[draw,->] (a3) -- (b1);
        \path[draw,->] (a3) -- (b2);
        \path[draw,->] (a3) -- (b3);
        \path[draw,->] (a3) -- (bk);
    \node[below=2em of a3] (adots) {$\vdots$};
    \node[neurons,below=2em of adots] (ak) {};
        \path[draw,->] (ak) -- (b1);
        \path[draw,->] (ak) -- (b2);
        \path[draw,->] (ak) -- (b3);
        \path[draw,->] (ak) -- (bk);
        
    \node[input,left=5em of a] (x2) {$x_2$};
        \path[draw,->] (x2) -- (a1);
        \path[draw,->] (x2) -- (a2);
        \path[draw,->] (x2) -- (a3);
        \path[draw,->] (x2) -- (ak);
    \node[input,above=2em of x2] (x1) {$x_1$};
        \path[draw,->] (x1) -- (a1) ;
        \path[draw,->] (x1) -- (a2);
        \path[draw,->] (x1) -- (a3);
        \path[draw,->] (x1) -- (ak);
    \node[below=2em of x2] (xdots) {$\vdots$};
    \node[input,below=2em of xdots] (xn) {$x_n$};
        \path[draw,->] (xn) -- (a1);
        \path[draw,->] (xn) -- (a2);
        \path[draw,->] (xn) -- (a3);
        \path[draw,->] (xn) -- (ak);
        
    \node[left=0.2em of ak] (braceL) {};
    \node[right=0.2em of bk] (braceR) {};
    \draw [
    thick,
    decoration={
        brace,
        mirror,
        raise=0.7cm
    },
    decorate
    ] (braceL) -- (braceR) node[pos=.5,yshift=-3em] {skryté vrstvy}; 
    
    \node[below=2.5em of xn] {vstupní vrstva};
    
    \node[below=2.5em of om] {výstupní vrstva};
    
\end{tikzpicture}}  % chktex 31
\end{center}
}

\title{Neuronové sítě a hluboké učení}
\author{Jindřich Matuška}
\institute[FI MU]{Faculty of Informatics, Masaryk University}
\date{12.\ prosince 2024}

\AtBeginSection[]
{
    \begin{frame}
        \frametitle{Obsah}
        \tableofcontents[currentsection]
    \end{frame}
}

\begin{document}

\frame{\titlepage}

\frame{
    \frametitle{Čas na odpovědníky}
}

\frame{
    \frametitle{Kdo už viděl \ldots}

    Zvedněte:
    \begin{itemize}
        \item 1 ruku, pokud jste danou věc viděli,
        \item 2 ruce, pokud jste s věcí pracovali
    \end{itemize}
    \vspace{1em}

    \pause

    \begin{itemize}
        \item Perceptron, \pause Neuronová síť, \pause Konvoluční síť \pause
        \item Vnitřní potenciál $\xi$, \pause Prahová funkce, \pause ReLU, \pause Sigmoida \pause
        \item Chybová funkce $E$, \pause Kvadratická chyba \pause 
        \item Derivace, \pause Parciální derivace, \pause Diferenciál
    \end{itemize}
}

\begin{frame}
    \frametitle{Obsah}
    \tableofcontents
\end{frame}

\begin{section}{Struktura a výpočet neuronové sítě}

\frame{
	\nndemo{}
}

\frame{
	\frametitle{Perceptron}

	Vnitřní potenciál perceptronu
	\[\xi_i = -w_{0,i} + \sum_j w_{j,i}\cdot a_j\]
	Výstup jednotky ($g_i$ je aktivační funkce jednotky $i$)
	\[a(i) = g_i(\xi_i)\]
}

\frame{
	\frametitle{Aktivační funkce}

	Prahová funkce
	\[f(x) = \begin{cases}
		1 & x \geq 0\\
		0 & \text{jinak}
	\end{cases}\]

	ReLU
	\[f(x) = \begin{cases}
		x & x \geq 0\\
		0 & \text{jinak}
	\end{cases}\]

	Sigmoida
	\[\sigma(x) = \frac{1}{1+e^{-x}}\]
}

\frame{
	\frametitle{Příklad 11.1.1 a), b)}

Uvažte následující funkce $f$, $g$ a $h$.
\begin{center}
\begin{tikzpicture}
\begin{axis}[xmin=-0.5,ymin=-0.2,xmax=1.5,ymax=1.2, samples=20,
    legend cell align=left,
    legend pos=north west,
    width=0.5\textwidth,
    height=0.4\textwidth
%    legend style={draw=none}
    ]
  \addplot[blue, thick, domain=-0.5:0] {0 };
  \addplot[blue, thick, domain=0:1] {1 };
  \addplot[blue, thick, domain=1:1.5] {0 };

  \draw [draw=blue, fill=white, thick] (axis cs: 0, 0) circle (2.0pt);
  \draw [draw=blue, fill=blue, thick] (axis cs: 0, 1) circle (2.0pt);
  \draw [draw=blue, fill=blue, thick] (axis cs: 1, 1) circle (2.0pt);
  \draw [draw=blue, fill=white, thick] (axis cs: 1, 0) circle (2.0pt);
\end{axis}
  \node at (1.7, -0.78) {Funkce $f$};
\end{tikzpicture}
\hfil\begin{tikzpicture}
\begin{axis}[xmin=-1.2,ymin=-0.2,xmax=1.2,ymax=1.2, samples=20,
    legend cell align=left,
    legend pos=north west,
    width=0.5\textwidth,
    height=0.4\textwidth
%    legend style={draw=none}
    ]
  \addplot[blue, thick, domain=-1.2:0] {1+x };
  \addplot[blue, thick, domain=0:1.2] {1-x };
\end{axis}
  \node at (1.7, -0.78) {Funkce $g$};
\end{tikzpicture}
\end{center}

Aplikací operací skládání funkcí, násobení konstantou, přičítání konstanty a sčítání

\begin{enumerate}[a)]
    \item$\bigstar$ a pomocí prahové funkce vyjádřete funkci $f$,
    \item$\bigstar$ a pomocí ReLU vyjádřete funkci $g$,
\end{enumerate}
}

\frame{
	\frametitle{Příklad 11.1.3}

	\begin{center}
	\scalebox{0.6}{
	\begin{tikzpicture}

		\node[neurons] (center) {$y_1$};
		\node[right of=center] (right) {};
			\path[draw,->] (center) -- (right);
		\node[neurons,left=8em of center] (n2) {$n_2$};
			\path[draw,->] (n2) -- (center) node [weights, pos=.65] {10};
		\node[neurons,above=4.5em of n2] (n1) {$n_1$};
			\path[draw,->] (n1) -- (center) node [weights, pos=.7] {10};
		\node[neurons,below=4.5em of n2] (n3) {$n_3$};
			\path[draw,->] (n3) -- (center) node [weights, pos=.7] {10};
			
		\node[input,left=11em of n2] (2) {$x_2$};
			\path[draw,->] (2) -- (n1) node [weights, pos=.73, above] {10};
			\path[draw,->] (2) -- (n2) node [weights, pos=.75] {0};
			\path[draw,->] (2) -- (n3) node [weights, pos=.73, below] {-10};
		\node[input,above=3.5em of 2] (1) {$x_1$};
			\path[draw,->] (1) -- (n1) node [weights, pos=.5] {0};
			\path[draw,->] (1) -- (n2) node [weights, pos=.85, above] {-10};
			\path[draw,->] (1) -- (n3) node [weights, pos=.81, right] {10};
		\node[input,below=3.5em of 2] (3) {$x_3$};
			\path[draw,->] (3) -- (n1) node [weights, pos=.81, right] {-10};
			\path[draw,->] (3) -- (n2) node [weights, pos=.85, below] {10};
			\path[draw,->] (3) -- (n3) node [weights, pos=.5] {0};
	\end{tikzpicture}}  % chktex 31
	\end{center}

	Aktivační funkce je sigmoida $\sigma$, prahová váha $5$.
	\begin{enumerate}[a)]
		\item Jaké budou výstupy sítě pro vstupy $(0,0,0)$, $(1,0,1)$, $(1,1,1)$, $(-1, 0, 1)$, $(0.5, 0.4, 0.6)$? 
		\item Napište jednoduchý program, který pro zadaný vstup spočítá výstup sítě.
		\item Co síť počítá? Nalezněte co nejstručnější slovní charakteristiku.
		\item Stačila by pro takový výpočet neuronová síť bez skrytých vrstev?
	\end{enumerate}
}

\end{section}

\begin{section}{Učení neuronové sítě}

\frame{
    \frametitle{Učení neuronové sítě}

    \begin{itemize}
        \item Data:
            \[D = \{(\vec x_1, \vec y_{exp, 1}, \ldots, (\vec x_p, y_{exp, p}))\}\]  % chktex 35
        \item Chybová funkce, kvadratická chyba:
            \[E = E(\vec y, \vec y_{exp}) = {(||\vec y - \vec y_{exp}||)}^2\]  % chktex 35
        \item Diferenciál a parciální derivace chybové funkce:
            \[\nabla E = \left(\frac{\partial E}{w_1}, \frac{\partial E}{w_2},\ldots \right)\]
    \end{itemize}
}

\frame{
    \frametitle{Online algoritmus pro učení neuronové sítě}

    Tréninkový set $D = \{(\vec{x}_1, \vec{y}_1), \ldots, (\vec{x}_p, \vec{y}_p))\}$\\
    Počítáme sekvenci vah $\vec{w}^{(0)}, \vec{w}^{(1)}, \vec{w}^{(2)}, \ldots$

    \begin{itemize}
        \item $\vec{w}^{(0)}$ je inicializován náhodně hodnotami okolo 0,
        \item $\vec{w}^{(t+1)} = \vec{w}^{(t)} - \alpha \cdot \nabla E(\vec y^{(t)}, \vec y_k)$,
    \end{itemize}

    kde
    \begin{itemize}
        \item $k = (t \mod p) + 1$,
        \item $0 < \alpha \leq 1$ je konstanta učení,
        \item $\vec y^{(t)}$ je výsledek vrácený sítí v iteraci $t$ na vstupu $\vec y_k$.
    \end{itemize}
}

\frame{
    \frametitle{Online perceptronový algoritmus}

    Tréninkový set $D = \{(\vec{x}_1, c_1), \ldots, (\vec{x}_p, c_p))\}$\\
    Počítáme sekvenci vah $\vec{w}^{(0)}, \vec{w}^{(1)}, \vec{w}^{(2)}, \ldots$
    \begin{itemize}
        \item $\vec w^{(0)}$ je inicializována náhodně s hodnotami okolo 0,
        \item $\vec w^{(t+1)} = \vec w^{(t)} - \alpha\cdot (C[\vec w^{(t)}](\vec x_k) - c_k) \cdot \tilde x_k$,
    \end{itemize}
    kde
    \begin{itemize}
        \item $k = (t \mod p) + 1$,
        \item $0 < \alpha \leq 1$ je konstanta učení.
    \end{itemize}
}

\frame{
    \frametitle{Příklad 11.2.1 (pokud budeme chtít)}

    Viz sbírka
}

\end{section}

\begin{section}{Zpracování obrazu a klasifikace}

\frame{
    \frametitle{Zpracování obrazu}

    Úlohy:
    \begin{itemize}
        \item klasifikace, odhalení objektů,
        \item generování obrazu, upscaling, stylizace obrazu
        \item (de)komprese
    \end{itemize}

    Velký objem dat

    Časté využití konvolučních vrstev --- hledání rysů v obrazu
}

\frame{
    \frametitle{Softmax}

    Transformace výstupu na pravděpodobnosti
    \[\text{softmax}(x_1, \ldots, x_n) = \left(\frac{e^{x_1}}{\sum_{i=1}^n e^{x_i}}, \frac{e^{x_1}}{\sum_{i=1}^n e^{x_i}},\ldots, \frac{e^{x_n}}{\sum_{i=1}^n e^{x_i}}\right)\]
}

\end{section}
\end{document}  % chktex 17
