\documentclass{beamer}

% chktex-file 1
% chktex-file 8
% chktex-file 9
% chktex-file 10

\usetheme[workplace=fi]{MU}
\usepackage[utf8]{inputenc}
\usepackage[
  main=czech,
  english
]{babel}
\usepackage[utf8]{inputenc}
\usepackage[T1]{fontenc}
\usepackage{csquotes}
\usepackage{booktabs}
\usepackage{amsmath}
\usepackage{multicol}

\newcommand{\RR}{\ensuremath{\mathbb{R}}}

\usepackage{tikz}
\usetikzlibrary{arrows, shapes.geometric}

\newlength{\nodesize}
\setlength{\nodesize}{1.5em}
\tikzset{and/.style = {shape=circle,draw,minimum size=\nodesize}}
\tikzset{or/.style = {shape=rectangle,draw,minimum size=\nodesize}}

\tikzset{min/.style = {shape=circle,draw,minimum size=\nodesize}}
\tikzset{max/.style = {shape=rectangle,draw,minimum size=\nodesize}}
\tikzset{rand/.style = {diamond,draw,minimum size=\nodesize}}

\tikzset{edge/.style = {->,> = latex'}}

% Allows for split by delimiter
% \getNth{string}{delimiter}{entry_no}
% Source: https://tex.stackexchange.com/questions/12810/how-do-i-split-a-string
\ExplSyntaxOn
\NewDocumentCommand{\getNth}{mmm}
  {
    % #1 string, #2 separator, #3 index
    \seq_set_split:Nnx \l_tmpa_seq { #2 } { #1 }
    \seq_item:Nn \l_tmpa_seq { #3 }
  }
\ExplSyntaxOff

\setbeamertemplate{enumerate item}{\alph{enumi})}  % chktex 9 chktex 10

\newcommand{\tictactoe}[2]{%
    \draw (#1.center) ++(-.5,-.5) -- +(0,1);
    \draw (#1.center) ++(-.1667,-.5) -- +(0,1);
    \draw (#1.center) ++(.1667,-.5) -- +(0,1);
    \draw (#1.center) ++(.5,-.5) -- +(0,1);
    \draw (#1.center) ++(-.5,-.5) -- +(1,0);
    \draw (#1.center) ++(-.5,-.1667) -- +(1,0);
    \draw (#1.center) ++(-.5,.1667) -- +(1,0);
    \draw (#1.center) ++(-.5,.5) -- +(1,0);
    \draw (#1.center) ++(-.3333,.3333) node {\getNth{#2}{;}{1}};
    \draw (#1.center) ++(0,.3333) node {\getNth{#2}{;}{2}};
    \draw (#1.center) ++(.3333,.3333) node {\getNth{#2}{;}{3}};
    \draw (#1.center) ++(-.3333,0) node {\getNth{#2}{;}{4}};
    \draw (#1.center) ++(0,0) node {\getNth{#2}{;}{5}};
    \draw (#1.center) ++(.3333,0) node {\getNth{#2}{;}{6}};
    \draw (#1.center) ++(-.3333,-.3333) node {\getNth{#2}{;}{7}};
    \draw (#1.center) ++(0,-.3333) node {\getNth{#2}{;}{8}};
    \draw (#1.center) ++(.3333,-.3333) node {\getNth{#2}{;}{9}};
}

\title{Zpracování přirozeného jazyka}
\author{Jindřich Matuška}
\institute[FI MU]{Faculty of Informatics, Masaryk University}
\date{14.\ listopadu 2024}

\AtBeginSection[]
{
    \begin{frame}
        \frametitle{Obsah}
        \tableofcontents[currentsection]
    \end{frame}
}

\begin{document}

\frame{\titlepage}

\frame{
    \frametitle{Čas na odpovědníky}
}

\begin{frame}
    \frametitle{Obsah}
    \tableofcontents
\end{frame}

\begin{section}{Předzpracování dat}

\frame{
    \frametitle{Zdroje dat}
    Odkud můžeme čerpat data?
    \pause
    \begin{itemize}
        \item Weby (HTML stránky --- Odstranění značek, nebo z nich lze vytáhnout více dat?)
        \item Knihy (Získání textu ze stránek. Lze vytáhnout více dat?)
        \item Sociální sítě (Můžeme? Anonymizace?)
    \end{itemize}
}

\frame{
    \frametitle{Předzpracování dat}
    
    \begin{itemize}
        \item Co je naším cílem?
        \item Co je pro náš cíl důležité, podstatné?
        \item Co z dat dokážeme vyčíst? Dokážeme vyčíst něco více než holý text?
        \item Jsou data dostatečně čistá? Je třeba je vyčistit? Lze je vyčistit?
        \item Anotace dat?
        \item V jaké formě budeme data vůbec ukládat?
    \end{itemize}
}

\end{section}

\begin{section}{Gramatiky}
    
    \frame{
        \frametitle{Bezkontextové gramatiky}
        \begin{itemize}
            \item Množina terminálních symbolů $\Sigma = \{a, b, c, \ldots\}$
            \item Množina neterminálních symbolů $N = \{A, B, C, \ldots\}$
            \item Speciální neterminál (kořen) $S \in N$
            \item Soubor pravidel $\subseteq N \times V^*$, kde $V = \Sigma\cup N$
                \begin{itemize}
                    \item neterminál $\to$ libovolný řetězec
                    \item Pokud neterminál uvozuje více pravidel, používáme svislítko $\mid$
                \end{itemize}
        \end{itemize}
    }

    \frame{
        \frametitle{Příklad}

        \begin{align*}
            S &\to NP\ VP,\\
            NP &\to Noun \mid Ad\ NP,\\
            VP &\to Verb,\\
            Noun &\to \text{dítě} \mid \text{člověk} \mid \text{kapsa},\\
            Adj &\to \text{starý} \mid \text{cestující} \mid \text{nové},\\
            Verb &\to \text{píše} \mid \text{sedí} \mid \text{mluví},
        \end{align*}
    }

    \frame{
        \frametitle{Syntaktický strom}

        \begin{itemize}
            \item Kořenem je kořen gramatiky $S$
            \item Listy jsou terminály
            \item Potomci každého uzlu jsou seřazení
            \item Pro každý uzel:
                \begin{itemize}
                    \item uzel je terminál, nebo
                    \item potomci uzlu jsou pravidlem gramatiky
                \end{itemize}
            \item Může existovat více různých odvození
        \end{itemize}
        \vspace{1em}

        Tvorba syntaktického stromu z věty se nazývá syntaktický analýza
    }

    \frame{
        \frametitle{Příklad}

        \begin{align*}
            S &\to NP\ VP,\\
            NP &\to Noun \mid Adj\ NP,\\
            VP &\to Verb,\\
            Noun &\to \text{dítě} \mid \text{člověk} \mid \text{kapsa},\\
            Adj &\to \text{starý} \mid \text{cestující} \mid \text{nové},\\
            Verb &\to \text{píše} \mid \text{sedí} \mid \text{mluví},
        \end{align*}

        \begin{itemize}
			\item[1.] \uv{cestující sedí}
			\item[2.] \uv{nové nové kapsa píše}
			\item[3.] \uv{starý člověk mluví}
        \end{itemize}
    }

	\frame{
		\frametitle{Příklad 9.2.1}

        Uvažte gramatiku s následujícími pravidly:
        \begin{align*}
            S &\to NP\; VP \mid VP,\\
            NP &\to Noun \mid NP\; Conj\; NP, \\
            VP &\to NP\; \textit{Esse}
        \end{align*}
        a následujícím lexikonem:
        \begin{align*}
            Noun &\to \text{Romulus} \mid \text{Remus} \mid  \text{Danubius} \mid \text{fratellus} \mid \text{fratelli} \mid \text{fluvius}, \\
            \textit{Esse} &\to \text{sum} \mid \text{est}  \mid \text{sunt} \mid \text{eram} \mid \text{erat} \mid \text{erant},\\
            Conj &\to \text{et}
        \end{align*}

        \begin{itemize}
            \item[a)] Rozhodněte, které z následujících vět lze v gramatice vygenerovat.
            \begin{itemize}
                \item[1.] \uv{Romulus et Remus fratelli erant}
                \item[2.] \uv{Remus et Danubius et Romulus sum}
                \item[3.] \uv{Danubius est fluvius}
            \end{itemize}
            \item[b)] Nalezněte ke každé větě její syntaktický strom, pokud existuje.
        \end{itemize}
	}

    \frame{
        \frametitle{Pokrytí vs.\ přesnost}

        Pro zamýšlený jazyk $L$ a gramatiku $G$ generující jazyk $L(G)$:
        \begin{itemize}
            \item Pokrytí $|L\cup L(G)|\over|L|$
            \item Přesnost $|L\cup L(G)|\over|L(G)|$
        \end{itemize}
    }

    \frame{
        \frametitle{Příklad 9.2.3}

		Uvažme gramatiku $G$ s následujícími pravidly
		\begin{align*}
			S &\to AAA \mid BA \mid AB \mid C,\\
			A &\to a, \\
			B &\to bA \mid Ab, \\
			C &\to BA \mid AB \mid cB \mid Bc 
		\end{align*}
		a jazyk $L$ vět délky 3 sestavených ze slov $a$, $b$, $c$, které obsahují slovo $b$ právě jednou.

		\begin{enumerate}[a)]
			\item Je gramatika jednoznačná, neboli existuje pro každou věta jazyka $L(G)$ odvoditelnou v $G$ právě jeden syntaktický strom?
			\item Jaké je pokrytí gramatiky $G$ vzhledem k zamýšlenému jazyku $L$?
			\item Jaká je přesnost gramatiky $G$ vzhledem k zamýšlenému jazyku $L$?
		\end{enumerate}
    }

\end{section}
\end{document}  % chktex 17
