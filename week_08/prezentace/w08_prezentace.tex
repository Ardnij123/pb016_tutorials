\documentclass{beamer}

% chktex-file 1
% chktex-file 8
% chktex-file 9
% chktex-file 10

\usetheme[workplace=fi]{MU}
\usepackage[utf8]{inputenc}
\usepackage[
  main=czech,
  english
]{babel}
\usepackage[utf8]{inputenc}
\usepackage[T1]{fontenc}
\usepackage{csquotes}
\usepackage{booktabs}
\usepackage{amsmath}
\usepackage{multicol}

\newcommand{\RR}{\ensuremath{\mathbb{R}}}

\usepackage{tikz}
\usetikzlibrary{arrows, shapes.geometric}

\newlength{\nodesize}
\setlength{\nodesize}{1.5em}
\tikzset{and/.style = {shape=circle,draw,minimum size=\nodesize}}
\tikzset{or/.style = {shape=rectangle,draw,minimum size=\nodesize}}

\tikzset{min/.style = {shape=circle,draw,minimum size=\nodesize}}
\tikzset{max/.style = {shape=rectangle,draw,minimum size=\nodesize}}
\tikzset{rand/.style = {diamond,draw,minimum size=\nodesize}}

\tikzset{edge/.style = {->,> = latex'}}

% Allows for split by delimiter
% \getNth{string}{delimiter}{entry_no}
% Source: https://tex.stackexchange.com/questions/12810/how-do-i-split-a-string
\ExplSyntaxOn
\NewDocumentCommand{\getNth}{mmm}
  {
    % #1 string, #2 separator, #3 index
    \seq_set_split:Nnx \l_tmpa_seq { #2 } { #1 }
    \seq_item:Nn \l_tmpa_seq { #3 }
  }
\ExplSyntaxOff

\setbeamertemplate{enumerate item}{\alph{enumi})}  % chktex 9 chktex 10

\newcommand{\tictactoe}[2]{%
    \draw (#1.center) ++(-.5,-.5) -- +(0,1);
    \draw (#1.center) ++(-.1667,-.5) -- +(0,1);
    \draw (#1.center) ++(.1667,-.5) -- +(0,1);
    \draw (#1.center) ++(.5,-.5) -- +(0,1);
    \draw (#1.center) ++(-.5,-.5) -- +(1,0);
    \draw (#1.center) ++(-.5,-.1667) -- +(1,0);
    \draw (#1.center) ++(-.5,.1667) -- +(1,0);
    \draw (#1.center) ++(-.5,.5) -- +(1,0);
    \draw (#1.center) ++(-.3333,.3333) node {\getNth{#2}{;}{1}};
    \draw (#1.center) ++(0,.3333) node {\getNth{#2}{;}{2}};
    \draw (#1.center) ++(.3333,.3333) node {\getNth{#2}{;}{3}};
    \draw (#1.center) ++(-.3333,0) node {\getNth{#2}{;}{4}};
    \draw (#1.center) ++(0,0) node {\getNth{#2}{;}{5}};
    \draw (#1.center) ++(.3333,0) node {\getNth{#2}{;}{6}};
    \draw (#1.center) ++(-.3333,-.3333) node {\getNth{#2}{;}{7}};
    \draw (#1.center) ++(0,-.3333) node {\getNth{#2}{;}{8}};
    \draw (#1.center) ++(.3333,-.3333) node {\getNth{#2}{;}{9}};
}

\title{Důkazové systémy a rezoluce}
\author{Jindřich Matuška}
\institute[FI MU]{Faculty of Informatics, Masaryk University}
\date{14.\ listopadu 2024}

\AtBeginSection[]
{
    \begin{frame}
        \frametitle{Obsah}
        \tableofcontents[currentsection]
    \end{frame}
}

\begin{document}

\frame{\titlepage}

\frame{
    \frametitle{Čas na odpovědníky}
}

\begin{frame}
    \frametitle{Obsah}
    \tableofcontents
\end{frame}

\begin{section}{Obecná rezoluce ve výrokové logice}

\frame{
    \frametitle{S čím pracujeme}
    Pracujeme s klauzulemi
    \begin{itemize}
        \item Množiny literálů $\{l_1, \ldots, l_n\}$
            \begin{itemize}
                \item Ekvivalentní zápis $\bigvee_{i\in\{1,\ldots, n\}} l_i$
            \end{itemize}
        \item Každý literál je výroková proměnná ($p$) nebo její negace ($\neg p$)
        \item Speciálně symbol $\square$ je prázdná klauzule (kontradikce)
        \item Množinu klauzulí chápeme jako jejich konjunkci
    \end{itemize}
}

\frame{
    \frametitle{Rezoluční pravidlo}
    Uvažujme klauzule $C_1 = \{p, l_1, \ldots, l_n\}$ a $C_2 = \{\neg p, {l'}_1, \ldots, {l'}_m\}$,
    kde $l_i$, ${l'}_j$ jsou literály.
    Jejich rezolventou je klauzule $C = \{l_1, \ldots, l_n, {l'}_1, \ldots, {l'}_m\}$.
    \vspace{1em}

    Rezolventa dvou klauzulí zachovává pravdivost v interpretaci.
    \[I(C_1) = 1 \wedge I(C_2)=1 \implies I(C) = 1\]
    Obrácená implikace nutně neplatí.
    \vspace{1em}

    \pause
    Proč? (Příklad 7.1.4)
}

\frame{
    \frametitle{Rezoluční důkaz $C$ z množiny klauzulí $S$}
    Konečná posloupnost klauzulí $C_1, \ldots, C_n$, kde $C_n = C$ a pro každé $C_i$:
    \begin{itemize}
        \item $C_i \in S$ nebo
        \item $C_i$ vzniklo aplikací rezolučního pravidla na předcházející klauzule
    \end{itemize}

    Pokud existuje takový rezoluční důkaz, řekneme že $C$ je rezolučně
    dokazatelná z $S$, zapisujeme $S \vdash_R C$.

    Pokud existuje důkaz $S \vdash_R \square$, nazveme tento důkaz vyvrácením $S$.

    Rezoluční důkazy zapisujeme často pomocí stromu.
    \vspace{2em}

    Důkaz klauzule $C=\{\neg q\}$ z množiny $S = \{\{p, r\}, \{\neg s, \neg r\}, \{\neg q, s\}, \{\neg p\}\}$
}

\frame{
    \frametitle{Příklad 7.1.1}

Obecnou rezolucí ukažte, že
\begin{itemize}
    \item[a)] formule $( p \vee q) \wedge (\neg{q} \vee \neg{r}) \wedge (\neg{p} \vee s) \wedge (\neg{s}) \wedge (r \vee s)$ je nesplnitelná,
    \item[b)] platí logické vyplývání $\{(\neg s\land t)\Rightarrow r, \neg s, \neg s \Rightarrow t\}\vDash r$.
\end{itemize}
}

\frame{
    \frametitle{Rezoluční vyvrácení}

    Pro množinu formulí $\mathcal T$ a formuli $\varphi$ platí $\mathcal T \models \varphi$,
    právě když $\mathcal T\cup \{\neg \varphi\}$ je nesplnitelná.
}

\end{section}

\begin{section}{SLD rezoluce ve výrokové logice}

\frame{
    \frametitle{SLD rezoluce}
    Lineární rezoluce na uspořádaných klauzulích s výběrovým pravidlem.
    \vspace{1em}

    Rezoluční strom SLD rezoluce je ve tvaru jedné větve.

    SLD strom zachycuje všechna možná vyhodnocení.
    \vspace{1em}

    Výběové pravidlo určuje na kterém literálu rezolvujeme.
    \begin{itemize}
        \item V tomto předmětu \textbf{vždy první literál} v klauzuli.
    \end{itemize}
    \vspace{1em}

    SLD rezoluce je úplná pro Hornovy klauzule
    \begin{itemize}
        \item každá klauzule má nejvýše 1 pozitivní literál
        \item \textit{fakta} --- pouze pozitivní literál, $[p]$
        \item \textit{pravidla} --- pozitivní literál a alespoň jeden negativní, $[p, \neg q, \neg r]$
        \item \textit{cíle} --- pouze negativní literály, $[\neg q, \neg r]$
    \end{itemize}
}

\frame{
    \frametitle{Příklad}

    Pomocí SLD rezoluce vyvraťte množinu
    \[\{ [\neg s, \neg u], [q, \neg r], [q, \neg t], [q], [r], [u, \neg q], [s]\}\]
    Nakreslete SLD strom systematického hledání řešení.
    Nakreslete rezoluční strom pro nej\-le\-věj\-ší úspěšnou větev SLD stromu.
}

\frame{
    \frametitle{Příklad 7.2.1}

    
Vysvětlete intuitivní význam jednotlivých typů Hornových klauzulí, tedy \textit{faktů}, \textit{pravidel} a \textit{cílů}.
}

\frame{
    \frametitle{Příklad 7.2.2}

SLD rezolucí ukažte nesplnitelnost množiny uspořádaných klauzulí $\{[q,\neg s,\neg r], [p,\neg q], [s], [r]\} \cup \{[\neg p,\neg s]\}$.
}

\end{section}

\begin{section}{Obecná rezoluce v predikátové logice}

    \frame{
        \frametitle{Obecná rezoluce v predikátové logice}

        Na formulích ve skolemově normální formě.

        Klauzule opět zapisujeme jako množiny literálů.

        Při rezoluci používáme substituci (nahrazení) za termy
        \begin{itemize}
            \item Nahradit můžeme obecnější term za specifičtější
            \item Proměnnou za libovolný term
        \end{itemize}
        \vspace{1em}

        Substituce
        \begin{itemize}
            \item Množina $\{x_1/t_1, \ldots, x_n/t_n\}$
            \item $x_i$ jsou proměnné, $t_i$ jsou termy
            \item Přejmenování proměnných, pokud všechny $t_i$ jsou proměnné
            \item Unifikátor $S$, pokud $|S\Phi| = 1$
        \end{itemize}

        \[S_1 = \{P(x, y), \neg Q(y, f(x))\}, S_2 = \{P(f(x)), P(y), P(f(f(c)))\}\]
        \[\Phi_1 = \{x/c, y/b\}, \Phi_2 = \{x/f(c), y/f(f(c))\}\]
    }

    \frame{
        \frametitle{Rezoluční pravidlo pro predikátovou logiku}

        Pro dvě klauzule \textbf{bez společných proměnných}
        \[\{P(\vec{t_1}), \ldots, P(\vec{t_k}), L_1, \ldots, L_n\} \text{ a }
        \{\neg P(\vec{t'_1}), \ldots, \neg P(\vec{t'_l}), L'_1, \ldots, L'_m\}\]
        a substituci $\Phi$, která je unifikátorem množiny
        \[\{P(\vec{t_1}), \ldots P(\vec{t_k}), P(\vec{t'_1}), \ldots P(\vec{t'_l})\}\]
        je jejich rezolventou
        \[\{L_1, \ldots, L_n, L'_1, \ldots, L'_m\}\Phi\]
    }

    \frame{
        \frametitle{Příklad 7.3.1}
        Nalezněte $S\Phi$ pro zadaná $S$ a $\Phi$.
        \begin{itemize}
            \item[a)] $S=\{P(x), Q(y)\}$, $\Phi=\{x/y, y/x\}$
            \item[b)] $S=\{P(c, x), \neg P(x, f(z)) \}$, $\Phi=\{f(z)/x, x/d\}$
            \item[c)] $S=\{P(x, y), P(c, z), P(c, c)\}$, $\Phi=\{x/y, y/z, z/c\}$
        \end{itemize}
    }

    \frame{
        \frametitle{Příklad 7.3.2}
        Nalezněte (nejobecnější) unifikátory následujících množin literálů ($a,b,c,d$ jsou konstanty).
        \begin{itemize}
            \item[a)] $S=\{P(x), P(y)\}$
            \item[b)] $S=\{P(x), Q(y)\}$
            \item[c)] $S=\{Q(x,x), Q(y,c)\}$
            \item[d)] $S=\{ Q(h(x,y), w), Q(h(g(v),a), f(v)), Q(h(g(v),a), f(b)) \} $
        \end{itemize}
    }

    \frame{
        \frametitle{Příklad 7.3.5}

        Uvažujte následující tvrzení:
        \begin{itemize}
            \item \uv{Existuje drak.}
            \item \uv{Draci spí nebo loví.}
            \item \uv{Když mají draci hlad, nespí.}
        \end{itemize}
        Rezolucí dokažte, že z uvedených tvrzení plyne \uv{Když má drak hlad, loví,} a to dvěma způsoby: (1) přímým odvozením závěru a (2) vyvrácením množiny obsahující předpoklady a negaci závěru. Zejména v řešení
        \begin{itemize}
            \item[a)] formalizujte uvedená tvrzení (předpoklady a závěr) v jazyce predikátové logiky,
            \item[b)] určete význam (interpretaci) jednotlivých predikátů,
            \item[c)] proveďte skolemizaci tvrzení a zapište je v klauzulárním tvaru a
            \item[d)] nakonec proveďte samotnou rezoluci.
        \end{itemize}
    }
\end{section}

\begin{section}{Dopředné a zpětné řetězení}

    \frame{
        \frametitle{Dopředné a zpětné řetězení}
        
        Báze znalostí --- Hornovy klauzule bez omezení
        \begin{itemize}
            \item Lze zapsat jako množina implikací a faktů
        \end{itemize}

        Dopředné řetězení --- začínáme u faktů, pomocí implikací odvozujeme další fakty

        Zpětné řetězení --- začínáme u dotazu, koukáme na možná pravidla odvození
    }

    \frame{
        \frametitle{Příklad 7.4.1}

        Proveďte algoritmus dopředného řetězení pro dotaz $E$, je-li zadána následující báze znalostí:\smallskip
        \begin{itemize}
            \item $(B\land D)\Rightarrow E$
            \item $A\Rightarrow B$
            \item $(A\land F)\Rightarrow D$
            \item $F\Rightarrow A$
            \item $C\Rightarrow B$
            \item $F$
        \end{itemize}
    }

\end{section}

\end{document}  % chktex 17
