\documentclass{beamer}

% chktex-file 1
% chktex-file 8

\usetheme[workplace=fi]{MU}
\usepackage[utf8]{inputenc}
\usepackage[
  main=czech,
  english
]{babel}
\usepackage[utf8]{inputenc}
\usepackage[T1]{fontenc}
\usepackage{csquotes}
\usepackage{booktabs}
\usepackage{amsmath}
\usepackage{multicol}

\newcommand{\RR}{\ensuremath{\mathbb{R}}}

\usepackage{tikz}
\usetikzlibrary{arrows, shapes.geometric}

\newlength{\nodesize}
\setlength{\nodesize}{1.5em}
\tikzset{and/.style = {shape=circle,draw,minimum size=\nodesize}}
\tikzset{or/.style = {shape=rectangle,draw,minimum size=\nodesize}}

\tikzset{min/.style = {shape=circle,draw,minimum size=\nodesize}}
\tikzset{max/.style = {shape=rectangle,draw,minimum size=\nodesize}}
\tikzset{rand/.style = {diamond,draw,minimum size=\nodesize}}

\tikzset{edge/.style = {->,> = latex'}}

% Allows for split by delimiter
% \getNth{string}{delimiter}{entry_no}
% Source: https://tex.stackexchange.com/questions/12810/how-do-i-split-a-string
\ExplSyntaxOn
\NewDocumentCommand{\getNth}{mmm}
  {
    % #1 string, #2 separator, #3 index
    \seq_set_split:Nnx \l_tmpa_seq { #2 } { #1 }
    \seq_item:Nn \l_tmpa_seq { #3 }
  }
\ExplSyntaxOff

\setbeamertemplate{enumerate item}{\alph{enumi})}  % chktex 9 chktex 10

\newcommand{\tictactoe}[2]{%
    \draw (#1.center) ++(-.5,-.5) -- +(0,1);
    \draw (#1.center) ++(-.1667,-.5) -- +(0,1);
    \draw (#1.center) ++(.1667,-.5) -- +(0,1);
    \draw (#1.center) ++(.5,-.5) -- +(0,1);
    \draw (#1.center) ++(-.5,-.5) -- +(1,0);
    \draw (#1.center) ++(-.5,-.1667) -- +(1,0);
    \draw (#1.center) ++(-.5,.1667) -- +(1,0);
    \draw (#1.center) ++(-.5,.5) -- +(1,0);
    \draw (#1.center) ++(-.3333,.3333) node {\getNth{#2}{;}{1}};
    \draw (#1.center) ++(0,.3333) node {\getNth{#2}{;}{2}};
    \draw (#1.center) ++(.3333,.3333) node {\getNth{#2}{;}{3}};
    \draw (#1.center) ++(-.3333,0) node {\getNth{#2}{;}{4}};
    \draw (#1.center) ++(0,0) node {\getNth{#2}{;}{5}};
    \draw (#1.center) ++(.3333,0) node {\getNth{#2}{;}{6}};
    \draw (#1.center) ++(-.3333,-.3333) node {\getNth{#2}{;}{7}};
    \draw (#1.center) ++(0,-.3333) node {\getNth{#2}{;}{8}};
    \draw (#1.center) ++(.3333,-.3333) node {\getNth{#2}{;}{9}};
}

\title{Predikátová logika}
\author{Jindřich Matuška}
\institute[FI MU]{Faculty of Informatics, Masaryk University}
\date{7. listopadu 2024}

\AtBeginSection[]
{
    \begin{frame}
        \frametitle{Obsah}
        \tableofcontents[currentsection]
    \end{frame}
}

\begin{document}

\frame{\titlepage}

\frame{
    \frametitle{Čas na odpovědníky}
}

\begin{frame}
    \frametitle{Obsah}
    \tableofcontents
\end{frame}

\begin{section}{Syntax}

\frame{
    \frametitle{Syntax \only<1>{výrokové}\only<2->{predikátové} logiky}
    \begin{itemize}
        \item Abeceda
            \begin{itemize}
                    \only<1>{{\color{red}
                \item Výrokové proměnné $\mathcal{P} = \{p, q, r, \ldots\}$
                    }}
                    \only<2->{{\color<2>{red}
                \item Symboly pro proměnné $\mathcal{V} = \{x, y, z, \ldots\}$
                \item Funkční symboly $f, g, h, \ldots$
                \item Predikátové symboly $P, Q, R, \ldots$
                }}
                \item Logické spojky $\neg, \vee, \wedge, \implies, \iff, \ldots$
                \item Pomocné symboly závorek $(, )$ \only<2->{{\color<2>{red} a kvantifikátorů $\forall, \exists$}}
            \end{itemize}
        \item<3-> Term
            \begin{itemize}
                \item Něco co můžeme vložit do predikátu
                \item Symboly pro proměnné $\mathcal V$
                \item Pro každý $n$-ární funkční symbol $f$ a termy $t_1, \ldots, t_n$,\\ $f(t_1, \ldots, t_n)$ je term
            \end{itemize}
        \item<3-> Jazyk predikátové logiky $\mathcal L$
            \begin{itemize}
                \item Množina funkčních a predikátových symbolů
            \end{itemize}
    \end{itemize}
}

\frame{
    \begin{itemize}
        \item Formule predikátové logiky
            \begin{itemize}
                \item Pro každý $n$-ární predikátový symbol a termy $t_1, \ldots, t_n$,\\
                    $P(t_1, \ldots, t_n)$ je (atomická) formule
                \item Jsou-li $\varphi, \psi$ formule, pak jsou formule i
                    $\neg (\varphi), (\varphi)\vee(\psi), (\varphi)\wedge(\psi), (\varphi)\implies(\psi), (\varphi)\iff(\psi), \ldots$
                \item Jsou-li $t_1$ a $t_2$ temy, pak $t_1 = t_2$ je formule
                \item Je-li $\varphi$ formule a $x$ proměnná, pak $\exists x (\varphi)$ a $\forall x (\varphi)$ jsou formule
                \item Množinu formulí výrokové logiky značíme $\mathcal{F}$
            \end{itemize}
    \end{itemize}
}

\frame{
	\frametitle{Výskyt proměnné}

	Výskyt proměnné $x$ ve formuli $\varphi$ je
		\begin{itemize}
		    \item \textit{vázaný}, existuje-li podformule $\varphi$, ozn. $\psi$, která obsahuje tento výskyt a začíná $\exists x$ nebo $\forall x$
            \item \textit{volný} v opačném případě
        \end{itemize}
	\vspace{1em}

    Formule, která nemá žádný volný výskyt proměnné se nazývá \textit{uzavřená} nebo též \textit{sentence}
}

\frame{
    \frametitle{Příklad 6.1.1}
	Ve formuli
		\[\varphi\equiv 2\ |\ x \Rightarrow \exists y(y\cdot 2 = x \vee \sin(x+y) > 1)\]
	kde všechny symboly mají obvyklý (matematický) význam,
    identifikujte (vč.\ arity, pokud to dává smysl) všechny
	\begin{itemize}
		\item[a)] proměnné (vč.\ jejich výskytů),  % chktex 9 chktex 10
		\item[b)] funkční a predikátové symboly.  % chktex 9 chktex 10
		\item[c)] termy,  % chktex 9 chktex 10
		\item[d)] logické spojky,  % chktex 9 chktex 10
		\item[e)] atomické formule,  % chktex 9 chktex 10
	\end{itemize}\medskip

	U funkčních a predikátových symbolů určete i jejich aritu.
}  % chktex 10

\frame{
    \frametitle{Příklad 6.1.2}
	Uvažujte jazyk $\mathcal L$ predikátové logiky obsahující  funkční a predikátové symboly zadané následující tabulkou:
    \begin{center}\begin{tabular}{l | l | l }  % chktex 44
    symbol & typ & arita \\\midrule
    $f $ & funkční & 3 \\
    $d $ & funkční & 0 \\
    $P $ & predikátový & 1 \\
    $Q $ & predikátový & 2 \\
    \end{tabular}\end{center}
    Rozhodněte, která z následujících slov jsou \textbf{termy} a \textbf{formule} jazyka $\mathcal L$:

    \noindent\begin{minipage}{0.35\linewidth}
\begin{itemize}
    \item[a)] $Q(d,d)$,  % chktex 9
    \item[b)] $z$,  % chktex 9 chktex 10
    \item[c)] $f(f(d))$,  % chktex 9 chktex 10
    \item[d)] $f(d,f(d,d,d),d)$,  % chktex 9 chktex 10
        \vspace{1em}\phantom{}
\end{itemize}
    \end{minipage}
    \begin{minipage}{0.6\linewidth}
\begin{itemize}
    \item[e)] $y=x$,  % chktex 9 chktex 10
    \item[f)] $\forall x (Q(d,d)=x)$,  % chktex 9 chktex 10
    \item[g)] $\forall y (f(f(x,y,z),d,d))$,  % chktex 9 chktex 10
    \item[h)] $\forall x (Q(f(f(d,d,d),y,z),f(z,d,y))\vee P(f(d,z,x)))$.  % chktex 9 chktex 10
\end{itemize}
    \end{minipage}
}  % chktex 10

\end{section}

\begin{section}{Sémantika}

\frame{
    \frametitle{Sémantika predikátové logiky}

    \begin{itemize}
        \item Interpretace $I$
            \begin{itemize}
                \item neprázdné univerzum (doména) $\mathcal D_{I}$
                \item $n$-ární relace $I(P) \subseteq \mathcal{D}_I^n$ pro každý $n$-ární predikátový symbol $P$
                \item Zobrazení $I(f): \mathcal{D}_I^n\to \mathcal{D}_I$ pro každý $n$-ární funkční symbol $f$
            \end{itemize}
        \item Valuace příslušící $I$
            \begin{itemize}
                \item Zobrazení $V: \mathcal V\to \mathcal D_I$
                \item Přiřazuje proměnným prvky univerza
            \end{itemize}
        \item Hodnota termu $t$ v interpretaci $I$ a valuaci $V$
            \begin{itemize}
                \item Prvek $|t|_{I, V}\in \mathcal D_I$
                \item $|t|_{I, V} = V(x)$ pro pokud je $t$ proměnná
                \item $|t|_{I, V} = I(f)(|t_1|_{I, V}, \ldots, |t_n|_{I, V})$ pokud $t=f(t_1, \ldots, t_n)$
            \end{itemize}
    \end{itemize}

    \[\forall x ((P(x, y) \vee P(f(x), g(x, y)))\implies Q(x, x, y))\]
}

\frame{
    \frametitle{Sémantika predikátové logiky 2}

    Formule $\varphi$ je pravdivá v interpretaci $I$ a valuaci $V$, značeno $\models_I^V \varphi$, právě když platí jedna z možností
    \begin{itemize}
        \item $\varphi \equiv P(t_1, \ldots, t_n)$ a $(|t_1|_{I, V}, \ldots, |t_n|_{I, V})\in I(P)$
        \item $\varphi \equiv t_1 = t_2$ a $|t_1|_{I, V} = |t_2|_{I, V}$
        \item \ldots Logické operátory \ldots
        \item $\varphi \equiv \forall x(\psi)$ a pro všechny prvky univerza $d\in\mathcal D$ platí $\models_I^{V'}\psi$, kde $V'$ vznikla z $V$ nahrazením obrazu $x$ prvkem $d$
        \item $\varphi \equiv \exists x(\psi)$ a existuje prvek univerza $d\in\mathcal D$, pro který platí $\models_I^{V'}\psi$, kde $V'$ vznikla z $V$ nahrazením obrazu $x$ prvkem $d$
    \end{itemize}
}

\frame{
    \frametitle{Pojmy}

    \begin{itemize}
        \item Formule $\varphi$ pravdivá v interpretaci $I$, značíme $\models_I \varphi$,
            jestliže je pravdivá v interpretaci $I$ pro libovolnou valuaci $V$.
        \item Formule $\varphi$ logicky pravdivá (tautologie), značíme $\models \varphi$,
            jestliže je pravdivá v každé interpretaci.
            % Ukázat, co všechno můžeme měnit pro pravdivou a logicky pravdivou
    \end{itemize}
}

\frame{
    \frametitle{Příklad 6.2.1}
    Ukažte, že existenční kvantifikátor $\exists$ lze v predikátové logice zavést jako syntaktickou zkratku, tj.\ ekvivalentními úpravami vyjádřete  $\exists x \varphi$ bez použití symbolu $\exists$.
    % Přes množinu?
}

\frame{
    \frametitle{Příklad 6.2.2 b)}  % chktex 9

    Nalezněte negace následujících formulí tak, aby se symboly $\neg$ nacházely
    výhradně bezprostředně před predikátovými symboly.
    \[\exists x((P(x) \wedge Q(x))\vee R(x))\]
}  % chktex 10

\frame{
    % Zde ukázat přístup "forall=oponent, exists=já"
    \frametitle{Příklad 6.2.3}
    Uvažte formuli $\varphi\equiv \forall x (P(x)\Rightarrow\exists y \neg Z(x,y))$. Pro každou z následujících interpretací rozhodněte, zda je v ní formule $\varphi$ pravdivá.
    \begin{itemize}
        \item[a)] $\mathcal D_{I} = \{ 0 \}$, $I(P)=\varnothing$, $I(Z)=\varnothing$  % chktex 9 chktex 10
        \item[b)] $\mathcal D_{I} = \mathbb Z$, $I(P)=\{1, 2, 3, \dots\}$, $I(Z)=\mathbb Z^2$  % chktex 9 chktex 10
        \item[c)] $\mathcal D_{I} = \mathbb Z$, $I(P)=\{1, 2, 3, \dots\}$, $I(Z)=\leq $  % chktex 9 chktex 10
        \item[d)] $\mathcal D_{I} = \{ 0 \}$, $I(P)=\{0 \}$, $I(Z)=\text{id}$\footnote{Relace identity obsahuje dvojice identických prvků, $\text{id}:=\{(x,y)\in \mathcal D_I^2; x=y\}$.}  % chktex 9 chktex 10
    \end{itemize}
}  % chktex 9 chktex 10

\frame{
    \frametitle{Příklad 6.2.7}
    Rozhodněte, v jakých interpretacích jsou pravdivé následující formule:
\begin{itemize}
    \item[a)] $ \forall x \exists y \exists z (((x=y) \vee (x=z)) \wedge (y \not = z)) $  % chktex 9 chktex 10
    \item[b)] $ \forall x \forall y \forall z ((Q(x,y)\land Q(y,z))\Rightarrow Q(x,z))$  % chktex 9 chktex 10
    \item[c)] $ \forall x (\neg Q(x,x)\land \exists y Q(x, y))$  % chktex 9 chktex 10
    \item[d)] $ \forall x (x\neq f(x)\land x=f(f(x)))$  % chktex 9 chktex 10
\end{itemize}}  % chktex 9 chktex 10

\frame{
    \frametitle{Příklad 6.2.8}
    Uvažujte jazyk $\mathcal L$ predikátové logiky obsahující binární funkční symbol  $+$ a unární predikátový symbol $K$.
    Uvažujte interpretaci (realizaci) $I$ jazyka $\mathcal L$, kde univerzum tvoří všechna celá čísla $\mathbb Z$, $+$ se realizuje jako sčítání a $K(x)$ jako \uv{$x$ je kladné}. Nalezněte
    \begin{enumerate}[topsep=1pt,itemsep=-1pt,partopsep=1ex]
        \item[a)] fomuli $\alpha(x)$, která je pravdivá v $I$ právě v těch valuacích $V$, že $V(x)=0$,  % chktex 9
        \item[b)] fomuli $\beta(x,y)$, která je pravdivá v $I$ právě v těch valuacích $V$, že $V(x)<V(y)$,  % chktex 9 chktex 10
        \item[c)] fomuli $\gamma(x,y)$, která je pravdivá v $I$ právě v těch valuacích $V$, že $V(x)=-V(y)$,  % chktex 9 chktex 10
        \item[d)] fomuli $\delta(x)$, která je pravdivá v $I$ právě v těch valuacích $V$, že $V(x)=1$.  % chktex 9 chktex 10
    \end{enumerate}
    Při definování formulí se můžete odvolat na formule, které jste již zavedli v předchozích bodech.
}  % chktex 10

\end{section}

\begin{section}{Normální formy}

\frame{
    \frametitle{Prenexová normální forma}
    Uzavřená formule $\varphi$ se nachází v \textit{prenexové normální formě (PNF)}, je-li tvaru
    \[Q_1 x_1\dots Q_n x_n \psi,\]
    kde $Q_i\in\{\forall,\exists\}$, $x_1,\dots,x_n$ jsou proměnné
    a formule $\psi$ je v konjunktivní normální formě
    (zejména tedy neobsahuje žádný kvantifikátor).
    \vspace{1em}

    Uzavřená formule $\varphi$ se nachází ve \textit{Skolemově normální formě}, nachází-li se v prenexové normální formě a obsahuje pouze obecné kvantifikátory $\forall$.
}

\frame{
    \frametitle{Algoritmus převodu do PNF (a Skolemovy NF)}

\begin{itemize}
    \item[I.] Eliminujeme zbytečné kvantifikátory.
    \item[II.] Přejmenujeme proměnné tak, aby u každého kvantifikátoru byla jiná proměnná.
    \item[III.] Eliminujeme jiné spojky než $\vee$, $\land$, $\neg$.
    \item[IV.] Přesuneme negaci až před samotné predikáty.
    \item[V.] Kvantifikátory přesuneme ven z jádra formule.
    \item[VI.] Jádro formule upravíme do KNF pomocí distributivních zákonů.
	\item[VII.] (Odstraníme existenční kvantifikátory zakódováním do předchozích všeobecně kvantifikovaných proměnných.)
\end{itemize}
}

\frame{
    \frametitle{Příklad 6.3.1 a)}  % chktex 9 chktex 10

    Převeďte následující formule do prenexové normální formy a proveďte
    skolemizaci.

    \[(\forall x \exists y Q(x,y) \vee \exists x \forall y P(x,y)) \wedge \neg \exists x \exists y \forall z R(x,y,z)\]
}  % chktex 9 chktex 10

\end{section}

\end{document}  % chktex 17
